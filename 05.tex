\tchapter{“……有的人没能在代表会议上捍卫布尔什维克!”党组织的选举——对党内各派别集团的新威胁}

基于上述内容,就不难理解И.Д.卡巴科夫为何要坚决阻挠斯大林采取的另一项重要的新举措——这项举措是一套全新的党组织领导人选举制度,其核心是允许对所有被提名候选人展开讨论与筛选,并对每位候选人进行无记名投票。一旦该制度推行,地方大员将无法确保把罗名册干部的职位留给自己属意之人;此外,在干部调动过程中,党组织内普通党员的意见也自此必须被纳入考量。\neditor{默认将注释视为作者注释,采用作者注释的格式。该条注释采用编者注释的格式。}

1937年2月至3月,在联共(布)中央委员会全会上,И.В.斯大林要求在党内生活中恢复“民主集中制”。他将接下来的工作目标表述如下:“秘密选举、毫无例外地否决候选人的权利和批评的权利——这就是你们自下而上进行检验的第二种手段。”“由此,我们有两种检验工作人员的途径:一种是从上级监察机关出发的自上而下的途径,另一种途径则是自下而上的途径,自下而上的监督。”斯大林教导指出,“而且,自下而上的监督有两种形式:通过积极分子进行监督,与此同时,领导干部要进行汇报;通过恢复我们党内的民主选举制来进行监督,这种情况下,党员有权否决任何一位候选人,尽可能多地进行批评,并迫使领导干部在广大党员群众面前进行汇报。”\nauthor{1937年联共(布)中央二月至三月全会的材料//《历史问题》,1995年,第11—12期,第15页。}

斯维尔德洛夫斯克州的首要负责人被迫接受了斯大林的这些新举措,然而,他们很快就遭遇了这些举措带来的“负面”后果。1937年2月至3月召开的联共(布)中央全会闭幕后,“党内民主”的火苗四处燃起,州党委俨然成为了一支团结友好的“救火队”。内部纷争不得不被搁置到一旁。

1937年2月,就在新的选举制度被引入前夕,州领导层萌生了撤换联共(布)斯维尔德洛夫斯克市列宁区党委第一书记,并提名原叶戈尔申诺大区党委书记尼古拉·瓦西里耶维奇·费德琴科担任这一职务的想法。费德琴科先是在区党代会上当选为区党委全体会议的参会代表,而后又在区党委全会上毫无阻碍地当选为区党委第一书记。从那时起,仅仅过去了三个月,至5月中旬,又一次党的代表会议召开,按照斯大林提出的最新措施,会议组建了新的区党委班子。

在讨论被提名候选人的阶段,出现了与费德琴科有关的严重问题。党代表会议的参会代表们得知,通常情况下,在叶戈尔申诺大区担任区委书记时,此人会“诅咒自我批评”,无视普通党员的意见,让“有问题的人”上位。书记们纷纷提出他在过往工作中的问题。在“审问”过程中,尼古拉·瓦西里耶维奇承认,多亏了他在共青团工作时的老相识、时任联共(布)彼尔姆市党委第一书记的И.Н.科尔苏诺夫(此人后来在“彼尔姆医疗委员会案件”中丧命),他才来到了乌拉尔。科尔苏诺夫把这位朋友安排到市党委,担任组织处长,也就是说,实际上让他成了自己的副手。与会者们提出的问题越来越多,而列席党代会的斯维尔德洛夫斯克市党委书记米哈伊尔·瓦西里耶维奇·库兹涅佐夫感觉到情况不妙,便走上讲台为州党委的漏洞辩解。“现在呈交的材料既不能证明费德琴科同志(对党)不忠诚,也不能证明他的工作能力有问题。”库兹涅佐夫对代表们说道。但是,斯大林式的民主使得不久前还不可能的事情成为了可能。“库兹涅佐夫同志,不能让党的代表会议面对这样一个事实啊:把代表们请来投票,却不想让他们关注表决的内容。”一位党代会与会者在大厅中的其他与会者的赞同声中说道。代表们开始一个接一个地对库兹涅佐夫秉持的立场表达不满。很明显,费德琴科被从候选人名单中剔除的可能性越来越大,这给州领导层和市领导层的计划判了死刑。晚上,К.Ф.普谢尼岑赶来帮助库兹涅佐夫,并开始努力推荐这位州领导层需要的候选人。最终,州党委主席团的成员们齐心协力做到了这一点。即便是在后来的无记名投票中,费德琴科也只得到了33张“反对”票,同时,得到了271张“赞成”票——在“被揭发”之前,州委书记们的权威一直很高\nauthor{斯维尔德洛夫斯克州社会组织文献中心,全宗号10,目录号3,卷宗号5,第48—90、508页;全宗号161,目录号6,卷宗号105,第7、15—16页。}。

另一个“热点”是下塔吉尔机车制造厂。在下塔吉尔机车制造厂的党员大会上,在选举工厂党委委员和市共产党员代表会议参会代表的过程中,共产党员们使厂长Г.З.帕夫洛茨基“落选”了。得知此事后,卡巴科夫通过电话联系下塔吉尔市党委书记Г.С.博加乔夫,要求他“找点理由”来否决投票结果,随后重新组织选举并把厂长“选上去”。不知出于何种原因,博加乔夫没有执行这些指示,然而,根据下塔吉尔党委书记的说法,在克拉斯诺乌拉尔斯克,“州主人”的类似指令却得到了执行\nauthor{出处同上,全宗号4,目录号15,卷宗号26,第56页。}。

卡巴科夫认为没有必要按照“斯大林在2—3月中央全会上做出的规定”来重新规划自己的工作,他继续推行在斯大林的改革发起之前就一直实施的那套政策。当提及他在图拉的仕途升迁时,他表现出了极大的积极性。诚然,这位无所不能的“州主人”有时也无法推举出他所需要的候选人,即便在旧的选举制度下也是如此。

例如,1936年1月,在第三次斯维尔德洛夫斯克市党代表会议召开期间,在选举市党委委员的时候,就发生了这样的事件。按照当时的惯例,市党委的代表(这次由乌拉尔工业学院党委书记М.Н.马尔戈维奇扮演这一角色)走上讲台,向党代表会议的与会代表们宣读了预先准备好的市委委员候选人名单,随后,将以公开投票的方式对这份名单进行表决。在这份名单中,除了其他人员之外,还有斯维尔德洛夫斯克市苏维埃副主席А.С.科尔涅夫——对于卡巴科夫来说,此人并不是外人。在市苏维埃,科尔涅夫负责住宅—市政和公共设施环境的整治工作,自然,由于其所负责的城市公用事业领域的状况,他受到了严厉批评。根据职责分工,科尔涅夫负责监管斯维尔德洛夫斯克市的工程建筑企业和工业联合企业。这位市苏维埃副主席让自己的兄弟В.С.科尔涅夫掌管斯维尔德洛夫斯克市的工业联合企业。后者把工业联合企业的产品当作自己的财产加以支配,在管理企业期间,总共给国家造成了超过100万卢布损失,还侵吞了工业联合企业的近1.5万卢布的产品和资金。除了兄弟之外,А.С.科尔涅夫还把自己的妻子安排到这家联合企业工作,而且不是安排在普通岗位,而是担任了玩具车间的主任。这座车间经营不善,但是,它的主任却可以毫无缘由地连续几周不来上班。

卡巴科夫和库兹涅佐夫并非偶然地将“自己人”安插到了这些职位上:正如后来查明的那样,工业联合企业是资金流入他们自己的腰包的渠道之一。例如,斯维尔德洛夫斯克市的财政部门向工业联合企业商店的一个活期账户划入2万卢布,而这些钱随后就被用于市委的各项开支。

当州检察院最终还是对工业联合企业的业务状况产生了兴趣的时候,监督部门响起了一阵电话铃声。市苏维埃主席尼古拉·尼基托维奇·米森科吓唬负责侦查的格拉西莫维奇说:“既然没有得到市苏维埃关于将科尔涅夫送交法庭审判的决定,你们有什么依据去查工业联合企业的案子?”\nauthor{出处同上,全宗号161,目录号1,卷宗号676,第271页;全宗号4,目录号15,卷宗号1,第43页;К.布哈林《市苏维埃的事务》//《乌拉尔工人报》,1937年4月14日。}

市领导层在满足自身需求时毫不吝啬钱财,却力求在城市的公共开支中节省资金。其中,А.С.科尔涅夫被指定负责在斯维尔德洛夫斯克举办航空日活动,根据初步估算,届时前来参加庆祝活动的城市居民将不会少于13万人。这样就出现了一个问题:在八月中旬,如何在举办如此大规模的活动时保障饮用水的供应。机场附近没有可供饮用的水井,斯维尔德洛夫斯克的贸易部门负责人罗森加乌兹表示他只能保证每人获得半升水,并建议用当地现有的(不适合饮用的)水源来补充不够用的水量。但是,代表斯维尔德洛夫斯克市卫生部门的А.А.捷连季耶夫强烈反对市党委常委的这项建议,提议从斯维尔德洛夫斯克运水过来,这自然需要额外的开支。主持市委常委会议的库兹涅佐夫粗暴地打断他说:“我们知道你是个什么样的医生!”,并且命令他闭嘴。库兹涅佐夫从下塔吉尔提携而来的罗森加乌兹的提议被采纳了\nauthor{斯维尔德洛夫斯克州社会组织文献中心,全宗号161,目录号6,卷宗号106,第40—41页。}。

尽管斯维尔德洛夫斯克市苏维埃的下属机构存在腐败现象,卡巴科夫仍然推动市苏维埃副主席(科尔涅夫)加入了市党委全体会议参会者的行列。要做到这一点并不容易:不久前召开的列宁区党代会上,在选举区党委委员时,科尔涅夫落选了。但是,这并没有让卡巴科夫感到不安,他向市党代表会议主席团成员表示,如果科尔涅夫不能当选为市委委员,那么他将把这视为联共(布)斯维尔德洛夫斯克市党委的一次政治失败。因此,主持市党代表会议的市委书记库兹涅佐夫试图耍花招:“我们以这份名单为基础怎么样?没有其它名单了吧?如果没有人提出异议,也许我们可以认为这份名单通过了吧?”

但出乎意料的是,奥尔忠尼启则区委第一书记Л.Л.阿维尔巴赫表示反对:“让我们按照全部规定对每个候选人进行投票表决吧。”

马尔戈维奇试图当场挽救这一局面,不过,结果却显得很荒唐。“我提议进行投票,但不统计票数,”学院党委书记说道。“好吧,那我们就逐个进行投票表决。”库兹涅佐夫不情愿地同意了,对马尔戈维奇的提议置之不理。

不出所料,在对科尔涅夫的候选人资格进行讨论时,会场上出现了僵局:一位党代会参会者站出来发言。他没有详述细节,只是宣称在А.С.科尔涅夫管辖的工业联合企业中存在违法和任人唯亲的现象,而且城市建设事业的情况也并非良好:在斯维尔德洛夫斯克建造的学校存在大量工程缺陷,总损失金额高达100万卢布。这位代表毫不讳言,在列宁区党代会上对科尔涅夫的讨论是他一手促成的,并且直截了当地总结道:“我认为科尔涅夫同志不能被选入市党委全体会议与会者的行列。”

究竟是谁有胆量如此公开且坚决地站出来反对И.Д.卡巴科夫所力挺的人呢?要知道,在斯维尔德洛夫斯克,科尔涅夫兄弟背后有谁撑腰,对任何人来说都不是什么秘密。

这位代表就是亚历山大·伊万诺维奇·多加多夫——自1905年起就加入了布尔什维克党的老党员。仅列举一下多加多夫履历表上的一些职务,就能回答“他为何有此胆量”的问题。20世纪20年代至30年代初,多加多夫担任了联共(布)中央组织局委员,还担任过其他高级领导职务,包括全联盟工会中央委员会第一书记和苏联最高国民经济委员会副主席的职位。在党的第十七次代表大会上,他当选为苏联人民委员会下属的苏维埃监察委员会委员,随后被任命为委员会驻斯维尔德洛夫斯克州专员。在斯维尔德洛夫斯克,仅有几个人能够在不太担心自身命运的情况下说出与“州领导人”观点的不同的看法,А.И.多加多夫就是其中之一\nauthor{苏联共产党、联共(布)、俄共(布)、俄国社会民主工党(布)中央委员会:历史—传记手册/Ю.В.戈里亚乔夫编,莫斯科,2005年,第196页;全俄共产党(布)第十七次代表大会……第681页;中央党务档案馆,全宗号161,目录号1,卷宗号676,第270—272页。}。

另一位“来自图拉的人”即联共(布)斯大林区委第一书记马克西姆·费奥多罗维奇·科索夫也跳出来为科尔涅夫辩护。但是,他的发言并不是太成功。这位区委书记先是笨拙地试图为科尔涅夫在人事方面的“错误”开脱,声称“干部是由其他组织挑选的”。随后,他又试图反驳多加多夫关于学校建设存在缺陷的说法:“……在斯维尔德洛夫斯克市,它(建筑工程企业)把学校建得相当不错——这是事实。”大厅里响起了一阵喧哗声:与会代表们对这种明显歪曲事实的言论感到愤怒。

没有人再站出来为科尔涅夫辩护了。代表们以公开投票(!)的方式对市苏维埃副主席是否具备进入市党委委员行列的候选人资格的问题给出了回应。

随后,卡巴科夫不得不将科尔涅夫从市苏维埃中调走。当然,卡巴科夫和库兹涅佐夫没有让这位来自图拉的亲信无事可做,而是悄悄地在斯维尔德洛夫斯克给他安置了一个“肥差”——他担任了涡轮机厂建设事务负责人的副手\nauthor{斯维尔德洛夫斯克州社会组织文献中心,全宗号161,目录号6,卷宗号105,第54—58页;目录号1,卷宗号676,第272—274页。}。

然而,在接下来的1937年,试图让科尔涅夫的顶头上司——斯维尔德洛夫斯克市苏维埃主席尼古拉·尼基季奇·米森科选入市党委委员行列的这一失败尝试引起了全联盟范围的反响。原因在于,这些事件是在2月至3月召开联共(布)中央全会的背景下发生的。

对于州领导层和市领导层来说,米森科是一个非常“合用”的人物。尽管这座城市实际上没有正常运转的排水系统和供水系统,污水被排放到城市的水体中,造成不卫生的状况,而且政府拨给公用事业建设的资金要么没有被使用,要么被挪作他用。1936年,莫斯科拨付300万卢布用于建设斯维尔德洛夫斯克的污水收集管道,然而工程本身只使用了80万卢布,计划修建6公里的管道,实际只建成了大约150米。1937年,中央又拨出了大约500万至600万卢布,用于修建同样的污水收集管道,但是,根据卡巴科夫的指示,这些钱被挪作其他用途,包括修建喷泉,而污水收集管道的建设则被搁置了。从联共(布)中央政治局委员А.А.安德烈耶夫在1937年5月召开的州党委全会上激动的发言中就能了解到当时作为州中心的这座城市的状况:“看看吧,这里做的这些事简直就是耻辱!只要看一眼,就能想象得出建设方面的情况:供水系统、排水系统等的建设情况。一切都被打乱了,什么都没有完工,一切都只停留在光秃秃的地基上,没有建成。在城里的街道上都没法走路。在这些年里,没有建成什么特别新的、重要的东西。1933年,我来过斯维尔德洛夫斯克,昨天我又仔细地看了看,乘车转了转。几乎没有建成什么新东西,没有做什么特别的事情,这座城市只是变得更脏了,更加杂乱无章了。”\nauthor{俄罗斯国家社会政治史档案馆,全宗号73,目录号2,卷宗号27,第12页。}

1937年,斯维尔德洛夫斯克市“装点”了159座未完工的建筑,为此耗费了近三千万卢布,却毫无成效。关于完成已搁置的建设工程的提议没有得到市领导层的支持,资金都被投入到新的建筑项目中去了\nauthor{斯维尔德洛夫斯克州社会组织文献中心,全宗号161,目录号6,卷宗号104,第209页。}。

但是,市苏维埃的领导人们会关注诸如供水系统、污水收集管道以及其它公用事业建筑项目这类“琐事”吗?他们忙于装修自己的办公室和接待室,据最粗略的估算,为此花费了超过9万卢布。

市苏维埃主席非常关注“大乌拉尔”酒店的工作。1936年底,他从莫斯科聘请了一个叫切尔内绍夫的人担任酒店和餐厅的经理,给他定下了不低的月薪——1500卢布,还提供了一套公寓,免除了市政服务费用,并且像对待前往苏联偏远地区的人员那样给他发放双倍的出差补贴。米森科给切尔内绍夫下达了一项任务,让他在莫斯科(而且只能是莫斯科)为“大乌拉尔”酒店挑选一名餐厅经理和两名有经验的酒店管理员,在斯维尔德洛夫斯克为他们提供免费的公寓和伙食,另外还要挑选一名值班管理员和三名女服务员。米森科还计划邀请一个吉卜赛合唱团和一支爵士乐队到餐厅来演出。但这些计划都未能实现:切尔内绍夫很快就露出了他那唯利是图的本性。在市苏维埃拿到了用于购置理发店设备的1.3万卢布后,切尔内绍夫在敖德萨随意购买各种没有任何账目的商品,总价值约3000卢布(据他自己说,他购买的东西价值6400卢布),然后就带着剩下的9700卢布消失了。不知出于什么原因,米森科拒绝将与切尔内绍夫有关的材料提交法庭,而是下令(只是不清楚用什么方式)从这个逃之夭夭的家伙那里追回短缺的款项\nauthor{出处同上,卷宗号105,第153—154、169页;К.布哈林《市苏维埃的事务》。}。

在市苏维埃主席权力象征中,联共(布)市委委员这一身份是不可或缺的一部分。有人原本预计米森科将被斯维尔德洛夫斯克市列宁区的党组织推选成为斯维尔德洛夫斯克市党代表会议的参会代表。然而,列宁区党代表会议的代表选举过程并未按照预先设定的剧本进行。尽管包括市委书记库兹涅佐夫等人在内的市党委付出了努力,但是,米森科的候选人资格在确定投票名单的阶段就被否决了。

于是,库兹涅佐夫下令将市苏维埃主席紧急“转到”另一个区的党代表会议——卡冈诺维奇区的党代表会议去参选。这一次,为了给米森科争取机会,区党委做了更充分的准备,对参会代表们进行了逐个游说。铁路党组织的代表被召集了两次,他们被劝说要按要求投票,并且对方毫不掩饰地表示这是上级机关的意愿。市委常委委员、党内干部处处长И.А.科尔米洛夫竭尽全力地进行运作。但是,所有努力再次付诸东流。

在又一次失败之后,库兹涅佐夫向斯大林区党委第一书记М.Ф.科索夫发出指示,要求无论如何都要让米森科在他所在区的党代表会议上当选为代表。库兹涅佐夫和科尔米洛夫两人一同前往斯大林区参加党代表会议。但是,这一次,幸运之神也没有眷顾市苏维埃主席及其庇护者们:党代表会议投票表决,反对米森科当选。

代表们质问如此拼命地要把米森科推选上去的库兹涅佐夫:为什么市委要固执地推举在其它区的党代会上已被否决的候选人呢?市委书记找到了说辞:“我们想在所有的党代会上了解大家对他的看法。”而市苏维埃主席却认为所做的努力还不够。“布尔什维克市党委的领导们没能在党代会上为我辩护。”他毫不掩饰自己的懊恼之情。还有一次,在谈到自己的处境时,他想起了《圣经》里的情节,就像使徒彼得否认他与耶稣基督的关系那样\nauthor{斯维尔德洛夫斯克州社会组织文献中心,全宗号161,目录号6,卷宗号105,第82页;卷宗号154,第34—35页;全宗号10,目录号3,卷宗号5,第63页;М.克鲁戈洛夫《斯维尔德洛夫斯克市委“吸取教训”》//《真理报》,1937年3月3日。}。

3月3日,在《真理报》刊发《斯维尔德洛夫斯克市委“吸取了教训”》一文后,这段与试图将米森科“塞进”市党委委员行列有关的丑闻事件开始广为人知。尽管文章中描述的事件发生在实行新的党委选举制度的前夕,但是,《真理报》对这些事件的披露旨在说明当地存在违反“党内民主”原则和党委选举制度中选举制原则的情况\nauthor{《在自我批评和联系群众的旗帜下!》//《真理报》,1937年3月6日;《党的政治工作的改组》//《真理报》,1937年3月11日。}。

米森科的地位动摇了。联共(布)中央党监察委员会全权代表К.И.布哈林谈到了让他辞职的必要性。然而,尽管发生了重大丑闻,卡巴科夫和库兹涅佐夫却并不急于与米森科分道扬镳,理由是没有合适的人来取代他。作为回应,在当时,手中掌握着揭露米森科材料的布哈林找到了一种方法来把米森科占据的这个受庇护的职位从他的手中夺过来。在4月14日出版的《乌拉尔工人报》上,他发表了一篇访谈文章《市苏维埃的事务》,并使其在市苏维埃全会召开之际得以发表。文章谈到了市苏维埃领导们的腐败问题,首先就是米森科,还提到了他的政治错误(诚然,文中说得不是很清楚,也没有确凿的证据基础)。在文章结尾,布哈林抨击了市委(我们要记得,市党委第一书记正是由卡巴科夫本人兼任!),指责市委在领导市苏维埃和市基层党组织方面存在错误,要求米森科在苏维埃积极分子会议上就二月至三月召开的联共(布)中央全会的决议作报告。文章的最后一句话极具挑衅性:“这算什么,这难道不是联共(布)斯维尔德洛夫斯克市委领导人们的政治短视吗?”\nauthor{К.布哈林《市苏维埃的事务》。}

这些领导人们的反应没有让人久等。报纸一经刊发,卡巴科夫就立即下令马上召集州党委常委会议,讨论布哈林的这篇文章。会议上,除了那些试图在不与身为州委第一书记的卡巴科夫发生冲突的情况下维护表面上“党内民主”的人(如道路部门负责人沙赫吉尔丹、报纸编辑茹霍维茨基和州委工厂部部长博罗夫斯基),几乎所有州党委常委委员和候补委员都攻击了联共(布)中央党监察委员会全权代表。米森科“对骗子的纵容”实在太明显了,以至于无法为其辩护。然而,卡巴科夫和库兹涅佐夫却坚决不愿意承认自己“在政治上的短视”。此外,与市苏维埃中明显存在的腐败情况不同,在阐述这一“政治短视”时,布哈林的论据确实存在明显问题。会议通过的决议内容就证明了这一点:“州委常委没有发现责成米森科同志在苏维埃积极分子会议上就联共(布)中央全会决议作报告一事存在错误,更没有发现存在政治短视的情况。会议受理布哈林同志的声明,并接受他正在核查的与米森科同志长期庇护敌人这一问题相关的补充材料,将这些材料提交州委常委审议。”

然而,卡巴科夫明白,完全对这起丑闻保持沉默是不可能的,米森科这个人不得不被舍弃掉。卡巴科夫和库兹涅佐夫决定推举米森科的副手、30岁的福姆·尼基季奇·费奥伦科来接替他的职位,顺便指出,这个人也是在布哈林的文章被点名的人物之一。卡巴科夫亲自出席市苏维埃全会,因此,市苏维埃主席的更替进行得很顺利,好像是《乌拉尔工人报》上根本没有发表过那篇文章一样。米森科被留在联共(布)市委常委委员的位置上,被送去疗养院休假,而且……上面开始为他准备上伊谢特“卡巴科夫”工厂厂长的职位\nauthor{出处同上;斯维尔德洛夫斯克州社会组织文献中心,全宗号4,目录号15,卷宗号50,第51页;全宗号161,目录号6,卷宗号105,第82—83页;卷宗号104,第160页;全宗号295,目录号1,卷宗号80,第32—33、35页。}。

卡巴科夫认为没有必要隐瞒他对当地的一些党的官员的计划——当然,这些人都是他的心腹,而且可能会在无记名投票中落选。他向来到斯维尔德洛夫斯克的中央选举委员会巡视员承认他计划让这些人来组建州执行委员会的机构班子\nauthor{斯维尔德洛夫斯克州社会组织文献中心,全宗号161,目录号6,卷宗号104,第126页。}。这就意味着,按照卡巴科夫的打算,那些在选举中落选的区级和市级领导们应该昂首挺胸地离开地方权力机构,搬到州中心来,在这里,等待他们的是高级别的权力职务、设施齐全的公寓、额外的现金补贴以及其它干部特权带来的好处。卡巴科夫就是打算如此轻率地让斯大林为了清除党和国家机关中的那些最为腐败的官员而做出的努力付诸东流。

庇护者与被包庇的人之间的关系紧密地缠绕着斯维尔德洛夫斯克州的领导阶层,这是党内干部的物质富足和他们在社会上享有特权地位的基础。这些关系赋予了И.Д.卡巴科夫的“乌拉尔帝国”以稳固性和力量,在卡巴科夫被捕后,人们经常称他为“乌拉尔帮派之父”\nauthor{出处同上,卷宗号105,第283页。}。也正是因为如此,这位“乌拉尔的领袖”才会不惜一切代价,无论如何都要将那些对他个人忠心耿耿的人留在党和国家机构的关键岗位上。

